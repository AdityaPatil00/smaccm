\documentclass[times, 10pt]{article}
%\usepackage{latex8}
\usepackage{times}
\usepackage{pslatex}
\usepackage{epsfig}
\usepackage{url}

\usepackage{listings}

\usepackage{tabularx}

\lstdefinelanguage{aadl}
{morekeywords={aadlboolean,aadlinteger,aadlreal,aadlstring,access,all,and,
        annex,applies,binding,bus,calls,classifier,connections,constant,
        data,delta,device,end,enumeration,event,extends,false,features,flow,
        flows,group,implementation,in,inherit,initial,inverse,is,list,memory,
        mode,modes,none,not,of,or,out,package,parameter,path,port,private,
        process,processor,properties,property,provides,public,range,
        reference, check, component,forall,features,
        false,true,contained,has_property,refined,refines,requires,server,set,sink,source,
        subcomponents,subprogram,system,thread,to,true,type,units,value},
morecomment=[l]{--}}

\lstset{language=aadl,
        basicstyle=\scriptsize\sffamily,
        aboveskip=.1cm, % \smallskipamount, % \bigskipamount,
        belowskip=.1cm, % \smallskipamount, % \bigskipamount,
        abovecaptionskip=.1cm, % \smallskipamount, % \medskipamount,
        belowcaptionskip=.1cm, % \smallskipamount, % \bigskipamount,
        xleftmargin=.0cm,
        captionpos=b,
        tabsize=3}


\newcommand{\onemedfig}[3]{%
  \begin{figure}[htbp]
    \centerline{\epsfig{file=#1.pdf,width=.4\textwidth}}
    \caption{#2}
    \label{#3}
  \end{figure}
}



\newcommand{\onefullfig}[3]{%
  \begin{figure}[htbp]
    \centerline{\epsfig{file=#1.pdf,width=.7\textwidth}}
    \caption{#2}
    \label{#3}
  \end{figure}
}



\title{AGREE tutorial}

\author{\textsc{Julien Delange}\\
  Software Engineering Institute\\
  \texttt{jdelange@sei.cmu.edu}
}

\begin{document}

\maketitle

\section{Introduction}
This document is a tutorial to learn to use the AGREE language and its
associated toolset. This is not a user-manual that covers all aspect of the
language features, all these aspects are described in the AGREE 
user-manual\footnote{\url{https://github.com/smaccm/smaccm/tree/master/documentation/agree}}.
This document is a way to learn how to use the language and its associated tools
through several case studies.

\onemedfig{imgs/agree-usage}{AGREE menu in OSATE outline}{fig:agree-usage}

    \subsection{Use Analysis Tools}
    To use AGREE, model components must be annotated with AGREE annex
    subclauses. Then, invoking AGREE can be done by selecting the top-level
    system instance and make a right-click and select two options:
    \begin{enumerate}
        \item
            \textbf{Verify Single Layer}: analyze and verify only one depth
            of the component hierarchy.
        \item
            \textbf{Verify All Layers}: analyze the complete components
            hierarchy.
    \end{enumerate}

    \subsection{Limitations}
    When using AGREE, your models must enforce some constraints. There is the
    list of the constraints your model has to enforce:
    \begin{itemize}
        \item
            \textbf{Execution Order}: the execution order of the model is
            done in the order of the declaration of the subcomponents.
        \item
            \textbf{Multiple fanin} are \textbf{not} supported. In other words,
            an incoming feature can have only one incoming connection.
        \item
            \textbf{Top-level component} must have an AGREE subclause, even
            if you do not want to verify anything and want to validate the
            subcomponent. Hopefully, you can insert a dummy subclause like the
            one shown below.
    \end{itemize}

\begin{lstlisting}
system mysystem
annex agree {**
	guarantee "dummy" : true;
**};
end mysystem;
\end{lstlisting}


    \subsection{Understanding Analysis Results}

\onefullfig{imgs/agree-results}{AGREE Results View in OSATE}{fig:agree-results}

\end{document}
