\documentclass[times, 10pt]{article}
%\usepackage{latex8}
\usepackage{times}
\usepackage{pslatex}
\usepackage{epsfig}
\usepackage{url}

\usepackage{listings}

\usepackage{tabularx}

\lstdefinelanguage{aadl}
{morekeywords={aadlboolean,aadlinteger,aadlreal,aadlstring,access,all,and,
        annex,applies,binding,bus,calls,classifier,connections,constant,
        data,delta,device,end,enumeration,event,extends,false,features,flow,
        flows,group,implementation,in,inherit,initial,inverse,is,list,memory,
        mode,modes,none,not,of,or,out,package,parameter,path,port,private,
        process,processor,properties,property,provides,public,range,
        reference, check,
        component,forall,features,eq,assert,assume,guarantee,int,pre,
        false,true,contained,has_property,refined,refines,requires,server,set,sink,source,
        subcomponents,subprogram,system,thread,to,true,type,units,value},
morecomment=[l]{--}}

\lstset{language=aadl,
        basicstyle=\scriptsize\sffamily,
        aboveskip=.1cm, % \smallskipamount, % \bigskipamount,
        belowskip=.1cm, % \smallskipamount, % \bigskipamount,
        abovecaptionskip=.1cm, % \smallskipamount, % \medskipamount,
        belowcaptionskip=.1cm, % \smallskipamount, % \bigskipamount,
        xleftmargin=.0cm,
        captionpos=b,
        tabsize=3}


\newcommand{\onemedfig}[3]{%
  \begin{figure}[htbp]
    \centerline{\epsfig{file=#1.pdf,width=.4\textwidth}}
    \caption{#2}
    \label{#3}
  \end{figure}
}



\newcommand{\onefullfig}[3]{%
  \begin{figure}[htbp]
    \centerline{\epsfig{file=#1.pdf,width=.7\textwidth}}
    \caption{#2}
    \label{#3}
  \end{figure}
}



\title{AGREE tutorial}

\author{\textsc{Julien Delange}\\
  Software Engineering Institute\\
  \texttt{jdelange@sei.cmu.edu}
}

\begin{document}

\maketitle

\section{Introduction}
This document is a tutorial to learn to use the AGREE language and its
associated toolset. This is not a user-manual that covers all aspect of the
language features, all these aspects are described in the AGREE 
user-manual\footnote{\url{https://github.com/smaccm/smaccm/tree/master/documentation/agree}}.
This document is a way to learn how to use the language and its associated tools
through several case studies.

    \subsection{Examples location}
    All the examples used in this tutorial are available online on the public
    OSATE github repository\footnote{\url{https://github.com/osate/examples/}}.
    You can import the model into your workspace directly to reproduce the
    examples presented through this tutorial.


\onemedfig{imgs/agree-usage}{AGREE menu in OSATE outline}{fig:agree-usage}

    \subsection{Use Analysis Tools}
    To use AGREE, model components must be annotated with AGREE annex
    subclauses. Then, invoking AGREE can be done by selecting the top-level
    system instance and make a right-click and select two options:
    \begin{enumerate}
        \item
            \textbf{Verify Single Layer}: analyze and verify only one depth
            of the component hierarchy.
        \item
            \textbf{Verify All Layers}: analyze the complete components
            hierarchy.
    \end{enumerate}

    \subsection{Limitations}
    When using AGREE, your models must enforce some constraints. There is the
    list of the constraints your model has to enforce:
    \begin{itemize}
        \item
            \textbf{Execution Order}: the execution order of the model is
            done in the order of the declaration of the subcomponents.
        \item
            \textbf{Multiple fanin} are \textbf{not} supported. In other words,
            an incoming feature can have only one incoming connection.
        \item
            \textbf{Top-level component} must have an AGREE subclause, even
            if you do not want to verify anything and want to validate the
            subcomponent. Hopefully, you can insert a dummy subclause like the
            one shown below.
    \end{itemize}

\begin{lstlisting}
system mysystem
annex agree {**
	guarantee "dummy" : true;
**};
end mysystem;
\end{lstlisting}


    \subsection{Understanding Analysis Results}
    For each component, AGREE provides the following analysis:
    \begin{enumerate}
        \item
            \textbf{Contract Guarantees}
        \item
            \textbf{Contract Assumptions}
        \item
            \textbf{Contract Consistency}
    \end{enumerate}


\onefullfig{imgs/agree-results}{AGREE Results View in OSATE}{fig:agree-results}


\section{First AGREE model}
To understand AGREE basics, we will design a first model with some basics
validation contracts. In this example, we will design a system with a sender
and a receiver. The sender sends an integer through its outgoing interface.

The integration contract will then guarantee that the value sent through
the interfaces is bound (with a range between 0 and 100) and the one received
by the receiver has the same bound as well.

    \subsection{Defining guarantees}
    In the following listing, we show the guarantee contract in the component type
    (the section defined with the \texttt{guarantee} keyword). The component
    implementation defines the behavior: the integer starts at 1, is incremented at
    each \textit{tick} and eventually set to 10 when the value reached 90.

    \begin{lstlisting}
system sender
features
    dataout : out data port base_types::integer;
annex agree {**
    guarantee "data sent is between 0 and 100": dataout < 100 and dataout > 0;
**};
end sender;

system implementation sender.i
annex agree {**
    eq k : int = 1 -> if (pre(k) > 90) then 10 else pre(k) + 1;
    assert (dataout = k);
**};
end sender.i;
\end{lstlisting}



    \subsection{Defining assumptions}
    We also need to define components assumptions so that the analysis tool
    can check that they are consistent with the guarantees. In the following
    listing, we define the assumption for the receiver, specifying that the
    data received is bound within a range of 0 and 100.
\begin{lstlisting}
system receiver
features
	datain : in data port base_types::integer;
annex agree {**
	assume "data produced between 0 and 100": (datain < 100) and (datain > 0);
**};
end receiver;
\end{lstlisting}

    \subsection{Running the tool}
    Once both \texttt{assume} and \texttt{guarantee} are defined, we integrate
    the components and can start the analysis tool. The results are shown in
    figure \ref{fig:agree-results}.

    To show how AGREE can help you to investigate errors within your system, we
    will introduce an error in the contract. Let's change the \texttt{guarantee}
    of the sender component and specify that the data sent be within a range of
    0 to 150. The component type specification should then looks like the
    following.

    \begin{lstlisting}
system sender
features
    dataout : out data port base_types::integer;
annex agree {**
    guarantee "data sent is between 0 and 150": dataout < 150 and dataout > 0;
**};
end sender;
\end{lstlisting}

When invoking the analysis tool again, it reports that the assumptions of the 
\texttt{rcv} component are not met, as shown in figure \ref{fig:agree-results-ko}.
When a contract is not validated, AGREE can then provide a counter example that
details the different execution paths that lead to the validation error. To get
the trace, right click on the assumption/guarantee/consistency contract not met
and select the option to show a counter example. Counter examples can be shown
in different format: text-based, spreadsheet (with Excel or OpenOffice) or in
Eclipse, as shown in figure \ref{fig:counter-examples}.

\onefullfig{imgs/system1-result-ko}{AGREE Results View in OSATE - Contracts not
validated}{fig:agree-results-ko}

\onefullfig{imgs/system1-counter-example}{Counter Example in
Eclipse}{fig:counter-examples}



\end{document}
