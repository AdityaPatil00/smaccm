\documentclass[times, 10pt]{article}
%\usepackage{latex8}
\usepackage{times}
\usepackage{pslatex}
\usepackage{epsfig}

\usepackage{listings}

\usepackage{tabularx}

\lstdefinelanguage{aadl}
{morekeywords={aadlboolean,aadlinteger,aadlreal,aadlstring,access,all,and,
        annex,applies,binding,bus,calls,classifier,connections,constant,
        data,delta,device,end,enumeration,event,extends,false,features,flow,
        flows,group,implementation,in,inherit,initial,inverse,is,list,memory,
        mode,modes,none,not,of,or,out,package,parameter,path,port,private,
        process,processor,properties,property,provides,public,range,
        reference,refined,refines,requires,server,set,sink,source,
        subcomponents,subprogram,system,thread,to,true,type,units,value},
morecomment=[l]{--}}

\lstset{language=aadl,
        basicstyle=\scriptsize\sffamily,
        aboveskip=.1cm, % \smallskipamount, % \bigskipamount,
        belowskip=.1cm, % \smallskipamount, % \bigskipamount,
        abovecaptionskip=.1cm, % \smallskipamount, % \medskipamount,
        belowcaptionskip=.1cm, % \smallskipamount, % \bigskipamount,
        xleftmargin=.0cm,
        captionpos=b,
        tabsize=3}




\newcommand{\onefullfig}[3]{%
  \begin{figure}[htbp]
    \centerline{\epsfig{file=#1.pdf,width=.7\textwidth}}
    \caption{#2}
    \label{#3}
  \end{figure}
}



\title{RESOLUTE documentation}

\author{\textsc{Julien Delange}\\
  Software Engineering Institute\\
  \texttt{jdelange@sei.cmu.edu}
}

\begin{document}

\maketitle

\section{RESOLUTE annex overview}
RESOLUTE is an AADL annex language used to validate AADL components with specific constraints.
The language is supported by an analysis tool that can verify the constraints
again the model. The analysis tool is implemented within OSATE, the AADL toolset
reference.


\section{Adding RESOLUTE annex to AADL components}
Adding resolute to AADL components is done by using the AADL annex mechanism.


\onefullfig{imgs/resolute-selection}{Invoking the tool on a system instance}{fig:resolute-invoke}

\section{Verifying a theorem}


\onefullfig{imgs/resolute-result}{The Assurance Case View in OSATE}{fig:assurance-case-view}


\section{Functions list}
\begin{itemize}
    \item
	    \textbf{has\_property (namedelement, property): boolean} - the
        namedlement passed as argument has the property passed as the 2nd
        argument.
    \item
	    \textbf{property (namedelement, property): value} - returns
        the value of the property defined on the namedelement.
    \item
	    \textbf{has\_parent (namedelement): boolean} - returns
        true if the component has a parent.
    \item
	    \textbf{parent (namedelement): namedlement} - returns
        the parent of the namedelelement.
    \item
	    \textbf{name (namedelement): string} - returns the name of the
        namedlement.
    \item
	    \textbf{type (namedelement): namedelement} - returns the type
        of the component passed as argument.
    \item
	    \textbf{has\_type (namedlement): boolean} - returns true
        is the argument as an associated type. Typically, this function
        is used on feature to know if this is a data port, an event port, etc.
    \item
        \textbf{is\_of\_type (namedelement, namedelement): boolean} - check
        that the element passed as first argument has the type
        defined as the second argument or one of its type extension.
    \item
	    \textbf{has\_member (namedelement, string): boolean} - the namedelement
        passed as first argument has a member (subcomponent, feature)
        which name is defined by the second argument.
    \item
   '    \textbf{features (namedelement): set} - returns a set containing
        all the features of the namedlement.
    \item
	'   \textbf{connections (namedelement): set} - returns all
        the incoming or outgoing connections for the component.
    \item
	    \textbf{subcomponents (namedelement): set} - returns
        a set containing all the subcomponents of the namedelement.
    \item
        \textbf{source (connection): namedelement}  - returns
        the source (mostly a feature) of a connection.
    \item
        \textbf{destination (connection): namedelement} - returns
        the destination (most of the time, a feature) of a connection.
    \item
        \textbf{direction (namedelement): string} - returns a string
        that shows the direction of a namedelement.
    \item
        \textbf{is\_event\_port (namedelement): string} - returns
        true is the namedlement is an event port.
    \item
        \textbf{lower\_bound (range): value} - returns
        the value of the lower bound of a range
    \item
        \textbf{upper\_bound (range): value} - returns the value
        of the upper bound of a range.
    \item
        \textbf{member(value, set): boolean}
    \item
        \textbf{sum (set): value}
    \item
        \textbf{union (set, set): set}
    \item
        \textbf{intersect (set, set): set}
    \item
        \textbf{instance (namedelement): namedelement} - returns the instance
        component of the of declarative component passed as argument.
    \item
        \textbf{instances (namedlement): set} - returns all
        the instances of the declarative component passed as argument.
    \item
        \textbf{analysis}
    \item
        \textbf{receive\_error}
    \item
        \textbf{contain\_error}
    \item
        \textbf{propagate\_error (namedelement, string): boolean}
    \item
        \textbf{error\_state\_reachable(namedelement, string): boolean}
\end{itemize}


\end{document}
