\documentclass[times, 10pt]{article}
%\usepackage{latex8}
\usepackage{times}

\usepackage{listings}

\usepackage{tabularx}

\lstdefinelanguage{aadl}
{morekeywords={aadlboolean,aadlinteger,aadlreal,aadlstring,access,all,and,
        annex,applies,binding,bus,calls,classifier,connections,constant,
        data,delta,device,end,enumeration,event,extends,false,features,flow,
        flows,group,implementation,in,inherit,initial,inverse,is,list,memory,
        mode,modes,none,not,of,or,out,package,parameter,path,port,private,
        process,processor,properties,property,provides,public,range,
        reference,refined,refines,requires,server,set,sink,source,
        subcomponents,subprogram,system,thread,to,true,type,units,value},
morecomment=[l]{--}}

\lstset{language=aadl,
        basicstyle=\scriptsize\sffamily,
        aboveskip=.1cm, % \smallskipamount, % \bigskipamount,
        belowskip=.1cm, % \smallskipamount, % \bigskipamount,
        abovecaptionskip=.1cm, % \smallskipamount, % \medskipamount,
        belowcaptionskip=.1cm, % \smallskipamount, % \bigskipamount,
        xleftmargin=.0cm,
        captionpos=b,
        tabsize=3}





\title{RESOLUTE documentation}

\author{\textsc{Julien Delange}}\\
  Software Engineering Institute\\
  \texttt{jdelange@sei.cmu.edu}
}

\begin{document}

\maketitle

\section{Functions list}
\begin{itemize}
    \item
	    \textbf{has\_property}
    \item
	    \textbf{property}
    \item
	    \textbf{has\_parent}
    \item
	    \textbf{parent}
    \item
	    \textbf{name}
    \item
	    \textbf{type}
    \item
	    \textbf{has\_type}
    \item
        \textbf{is\_of\_type}
    \item
	    \textbf{has\_member}
    \item
   '    \textbf{features}
    \item
	'   \textbf{connections}

    \item
	    \textbf{subcomponents}

    \item
        \textbf{source}

    \item
        \textbf{destination}

    \item
        \textbf{direction}

    \item
        \textbf{is\_event\_port}
    \item
        \textbf{lower\_bound}
    \item
        \textbf{upper\_bound}
    \item
        \textbf{member}
    \item
        \textbf{sum}
    \item
        \textbf{union}
    \item
        \textbf{intersect}
    \item
        \textbf{instance}
    \item
        \textbf{instances}
    \item
        \textbf{analysis}
    \item
        \textbf{receive\_error}
    \item
        \textbf{contain\_error}
    \item
        \textbf{propagate\_error}
    \item
        \textbf{error\_state\_reachable}
\end{itemize}


\end{document}
