\documentclass[times, 10pt]{article}
%\usepackage{latex8}
\usepackage{times}
\usepackage{pslatex}
\usepackage{epsfig}
\usepackage{url}

\usepackage{listings}

\usepackage{tabularx}

\lstdefinelanguage{aadl}
{morekeywords={aadlboolean,aadlinteger,aadlreal,aadlstring,access,all,and,
        annex,applies,binding,bus,calls,classifier,connections,constant,
        data,delta,device,end,enumeration,event,extends,false,features,flow,
        flows,group,implementation,in,inherit,initial,inverse,is,list,memory,
        mode,modes,none,not,of,or,out,package,parameter,path,port,private,
        process,processor,properties,property,provides,public,range,
        reference, check, component,forall,features,
        false,true,contained,has_property,refined,refines,requires,server,set,sink,source,
        subcomponents,subprogram,system,thread,to,true,type,units,value},
morecomment=[l]{--}}

\lstset{language=aadl,
        basicstyle=\scriptsize\sffamily,
        aboveskip=.1cm, % \smallskipamount, % \bigskipamount,
        belowskip=.1cm, % \smallskipamount, % \bigskipamount,
        abovecaptionskip=.1cm, % \smallskipamount, % \medskipamount,
        belowcaptionskip=.1cm, % \smallskipamount, % \bigskipamount,
        xleftmargin=.0cm,
        captionpos=b,
        tabsize=3}


\newcommand{\onemedfig}[3]{%
  \begin{figure}[htbp]
    \centerline{\epsfig{file=#1.pdf,width=.4\textwidth}}
    \caption{#2}
    \label{#3}
  \end{figure}
}



\newcommand{\onefullfig}[3]{%
  \begin{figure}[htbp]
    \centerline{\epsfig{file=#1.pdf,width=.7\textwidth}}
    \caption{#2}
    \label{#3}
  \end{figure}
}



\title{RESOLUTE tutorial}

\author{\textsc{Julien Delange}\\
  Software Engineering Institute\\
  \texttt{jdelange@sei.cmu.edu}
}

\begin{document}

\maketitle

\section{Introduction}
This tutorial gives a tour of the functionality of RESOLUTE.
It does not provide a complete description of the language, which
is rather provided by the user-manual of RESOLUTE\footnote{Documentation
avaialble on
\url{https://github.com/juli1/smaccm-improvements/tree/sei-improvements/documentation/resolute}}. This tutorial assumes that you have a working installation of RESOLUTE
and that you are able to use the validation tool.

This tutorial uses several examples hosted on OSATE github
repository. You can get the examples on the OSATE example
repository (see \url{https://github.com/osate/examples/}) under
the directory \texttt{core-examples/resolute}.

\section{Verifying property definition}
This first example checks the definition of a property
on a component. The related file is \texttt{property\_verification.aadl}.
The instance model is shown in figure \ref{fig:property-verification}.

\onemedfig{imgs/property_verification}{System instance in the \texttt{property\_verification.aadl}}{fig:property-verification}

In this model, each \texttt{thread} has the property
\texttt{resoluteps::foobar} defined:
\begin{itemize}
    \item
        \textbf{t1} has a value of 10
    \item
        \textbf{t1} has a value of 20
\end{itemize}

We define a theorem (\texttt{have\_good\_foobar}, see
file \texttt{theorems.aadl} in the textual model) to check
that all threads have a value for the \texttt{resoluteps::foobar}
greater than 15.

The theorem works as follow:
\begin{enumerate}
    \item
        It retrieves all the \texttt{thread} components
        associated to the process
    \item
        It calls the theorem \texttt{check\_thread} on each component.
    \item
        The \texttt{check\_thread} checks that the value
        of the \texttt{resoluteps::foobar} is greater than 15.
\end{enumerate}

\begin{lstlisting}
have_good_foobar(p : component) <=
	** " Check threads in component " p **
    forall(t : thread). contained(t, p) => check_thread (t)

check_thread(t : thread) <=
  	** "The thread " t " as a foobar bigger than 15"  **
  		(has_property (t, resoluteps::foobar)) and 
  		(property (t, resoluteps::foobar) > 15)
\end{lstlisting}

When running resolute on the initial model, the model
is not validated because \textbf{t1} has
a value of 10. To be able to validate the model,
change the definition of t1 and associate a value greater
than 15. For example, changing the definition
of the task as follow will be sufficient to validate the model.

\begin{lstlisting}
t1 : thread t.i {resoluteps::foobar => 20;};
\end{lstlisting}

After changing the model, run the analysis again, the
model would be validated.

In this part, we introduce the following RESOLUTE concepts:
\begin{enumerate}
    \item
        Call to another RESOLUTE theorem
    \item
        Use the \texttt{forall} keyword
    \item
        Use the \texttt{has\_property} built-in function to check
        that a property is defined on a component.
    \item
        Use the \texttt{property} built-in function to get the value
        of a property.
\end{enumerate}



\section{Analyzing Connections}
Now that we have discussed the verification of property values
in the components, we will present how you can analyze connections.
In this example, we will check that each incoming port
has only one incoming connection. This is a validation
oen might want to validate in the model in order to ensure
that there is only one sender for each communication port.
This type of modeling restriction is required by some tools such
as AGREE\footnote{AGREE actually does not support multiple fan-in, such
a validation tool can then help designers to check compliance of their model
against AGREE constraints}.

\onemedfig{imgs/double-fanin-instance}{System
Instance}{fig:double-fanin-instance}

The file related to this example is \texttt{check\_fanin.aadl}.
The related instance model is shown in figure \ref{fig:double-fanin-instance}.
The theorem \texttt{no\_double\_fanin} retrieves
all components within the model and validate them
using the \texttt{has\_single\_fanin} theorem. This theorem
gets all the incoming component features in a set and checks
that they only have one connection.

\begin{lstlisting}
no_double_fanin() <=
	** " All incoming feature have only one connection" **
    forall (c : component) . true => has_single_fanin (c)
 

has_single_fanin (comp : component) <=
	** " All incoming feature have only one connection on " comp **
	forall (f : features (comp)) . (direction(f) = "in") => (length (connections (f)) = 1) 
\end{lstlisting}

The model is not validated (result shown in figure
\ref{fig:double-fanin-result}) because components \texttt{snd1}
and \texttt{snd2} are connected to \texttt{recv} through
the same incoming feature. In order to be able to validate the model,
one solution is to add a new feature on \texttt{recv} and connect
\texttt{snd1} and \texttt{snd2} to a single and distinct port.

\onefullfig{imgs/double-fanin-result}{Analysis result for the double fanin system}{fig:double-fanin-result}

In this part, we introduced the following RESOLUTE concepts:
\begin{enumerate}
    \item
        Call to another RESOLUTE theorem
    \item
        Use the \texttt{forall} keyword
    \item
        Use the \texttt{direction} built-in function (returns
        either \texttt{"in"}, \texttt{"out"} or \texttt{"inout"}).
    \item
        Use the \texttt{length} built-in function that returns
        the size of a set
\end{enumerate}

\section{Analyzing Connections Consistency}
Something one might want to check is the consistency of connections
in an architecture. For example, checking a characteristic (such as
a property value) on features of all connections or on the sending/receiving
component.

This new example defines several connected components with a property defined
on their features (the same property as in the first example,
\texttt{resoluteps::foobar}). We will then use RESOLUTE to check
that the value on the connection source is lower than
the value on the connection destination.

\onemedfig{imgs/checkdataflow-instance}{System
Instance}{fig:checkdataflow-instance}

The architecture of the system is shown in figure
\ref{fig:checkdataflow-instance} and we associated
the following value on the components interfaces:
\begin{itemize}
    \item
        The outgoing feature of \texttt{snd} has a value of 5
    \item
        The incoming feature of \texttt{mid} has a value of 4
    \item
        The outgoing feature of \texttt{mid} has a value of 5
    \item
        The outgoing feature of \texttt{snd} has a value of 10
\end{itemize}

In order to check the architecture, we used theorem \texttt{minfoobarvalue}
defined in \texttt{theorems.aadl}. The theorem get all the outgoing
and incoming connections related to the component such as their source
and destination have the property \texttt{resolute::foobar} defined.
Then, for the source and the destination, the theorem checks that the 
property value on the source is lower than the property on the destination.

\begin{lstlisting}
  minfoobarvalue(c : component) <=
    ** " Check the minimum foobar for features on " c  **
    forall (conn : connections (c)) . has_property (source(conn), resoluteps::foobar) and
	                                  has_property (destination(conn), resoluteps::foobar) => 
	(property(source(conn), resoluteps::foobar) < property(destination(conn), resoluteps::foobar))
\end{lstlisting}

Considering this constraint, the model will not be validated because
in connection between \texttt{snd} and \texttt{mid}, the property value
associated with the source is bigger than the one associated with the
destination. Changing the property value on the source is then sufficient to
validate this constraint.


In this part, we introduced the following RESOLUTE concepts:
\begin{enumerate}
    \item
        Get the source and destination of a connection
    \item
        Get the property value associated with a feature
\end{enumerate}


\section{Checking Compliance of an Architecture}
For the last part of this tutorial, we will show how to check
the structure of an architecture using RESOLUTE. To illustrate
that, we will check the compliance of a model against the modeling
guidelines to design ARINC653 systems. The AADL ARINC653 annex
defines these rules. In a nutshell, there are the main guidelines:
\begin{itemize}
    \item
        Each AADL \texttt{process} must be bound to an AADL \texttt{virtual processor}
    \item
        Each AADL \texttt{virtual processor} must be contained in an AADL \texttt{processor}
    \item
        An AADL \texttt{processor} must define its configuration using
        appropriate properties (health monitoring, scheduling, etc.)
    \item
        Each AADL \texttt{process} must be bound to an AADL \texttt{memory} (a
        memory segment),
        which it itself contained in another AADL \texttt{memory} (the physical
        memory)
    \item
        Communication interfaces (\texttt{event data port} or \texttt{data
        port}) must define their requirements (sampling period, queueing
        protocol, etc.)
\end{itemize}

We plan to validate these rules against a model introduced in the ARINC653
annex. The graphical representation of the model is shown in
figure \ref{fig:arinc653-instance}.

\onefullfig{imgs/arinc653-instance}{System
Instance}{fig:arinc653-instance}

In order to validate the architecture compliance with the ARINC653 modeling
guidelines, we define several theorems. The top-level theorem
\texttt{check\_arinc653\_compliance} works as follow:
\begin{enumerate}
    \item
        Check the compliance of the architecture with
        the modeling guidelines to represent ARINC653 partitions
        by calling\texttt{check\_arinc653\_processes}. This theorem
        performs the following actions for each AADL \texttt{process}:
        \begin{enumerate}
            \item
                Call \texttt{check\_arinc653\_process\_memory} to check
                that the AADL \texttt{process} is bound to an AADL
                \texttt{memory}. It also calls
                \texttt{check\_arinc653\_memory\_segment} on the memory to make
                sure that the memory associated with the process is contained
                in a top-level (physical) memory.

            \item
                Call \texttt{check\_interfaces} to check that all communication
                requirements are defined using AADL properties:
                \begin{itemize}
                    \item
                        AADL \texttt{data port} define the property
                        \texttt{ARINC653::Sampling\_Refresh}
                    \item
                        AADL \texttt{event data port} define the property
                        \texttt{ARINC653::Queueing\_Discipline}
                \end{itemize}

            \item
                Call \texttt{check\_tasks} to check the requirements of AADL
                \texttt{thread} components contained in AADL \texttt{process}
                components. Mostly, the \texttt{check\_tasks} check the
                interfaces for the \texttt{thread} components.
            \item
                Call \texttt{check\_arinc653\_process\_virtual\_processor} to check
                that the AADL \texttt{process} under validation is bound
                to an AADL \texttt{virtual processor} component.
        \end{enumerate}
    \item
        Check the compliance of the architecture with
        the modeling guidelines to represent ARINC653 modules
        by calling\texttt{check\_arinc653\_processors}. This theorem
        performs the following actions for each AADL \texttt{processor}:
        \begin{enumerate}
            \item
                Check that the AADL \texttt{processor} defines
                the properties to define its scheduling policy
                by calling the \texttt{check\_arinc653\_processor\_scheduling}
                theorem. This theorem checks that the scheduling-related
                properties are correctly defined on the component
                (\texttt{ARINC653::Module\_Schedule} and 
                \texttt{ARINC653::Module\_Major\_Frame})
            \item
                Check that the AADL \texttt{processor} defines
                the properties to define its health monitoring policy
                by calling the \texttt{check\_arinc653\_processor\_hm}
                theorem. This theorem checks that the health-monitoring related
                properties are correctly defined on the component
                (\texttt{ARINC653::HM\_Error\_ID\_Levels} and 
                \texttt{ARINC653::HM\_Error\_ID\_Actions})
        \end{enumerate}
    \item
        Check the compliance of the architecture with
        the modeling guidelines to represent ARINC653 partition runtime
        by calling\texttt{check\_arinc653\_virtual\_processors}.
        For each AADL \texttt{virtual processor}, this theorem checks
        that it is contained in an AADL \texttt{processor} component but also
        defines appropriate configuration properties
        (\texttt{ARINC653::Partition\_Identifier} and
        \texttt{ARINC653::Partition\_Name}).
\end{enumerate}

When trying to validate the model, the validation fails because the health
monitoring properties are not declared (they are commented in the model).
Once the properties are correctly defined, the model is validated.


In this part, we introduced the following RESOLUTE concepts:
\begin{enumerate}
    \item
        Get the resource (processor or memory) bound to a component (built-in
        function \texttt{is\_bound\_to}
    \item
        Verify the containment rules of components (built-in function
        \texttt{parent})
    \item
        Call many RESOLUTE statements within several theorems
\end{enumerate}


\begin{lstlisting}
check_arinc653_compliance () <=
	** "Check compliance of the model with ARINC653 annex" **
	check_arinc653_processes() and
	check_arinc653_processors() and
	check_arinc653_virtual_processors ()


--
-- Procesor checks
--
    
check_arinc653_processors () <=
	** "Check compliance of the processors" **
	forall (cpu : processor) . true => check_arinc653_procesor (cpu)

check_arinc653_procesor (cpu : processor) <=
	** "Check compliance of processor " cpu **
	 check_arinc653_procesor_scheduling (cpu) and
	 check_arinc653_procesor_hm (cpu)
		 
check_arinc653_procesor_hm (cpu : processor) <=
	** "Check compliance of processor " cpu " for Health-Monitoring properties" **
	 has_property (cpu, ARINC653::HM_Error_ID_Levels) and
	 has_property (cpu, ARINC653::HM_Error_ID_Actions)

check_arinc653_procesor_scheduling (cpu : processor) <=
	** "Check compliance of processor " cpu " for scheduling properties" **
	 has_property (cpu, ARINC653::Module_Schedule) and
	 has_property (cpu, ARINC653::Module_Major_Frame)
	
	
-- 
-- Virtual Processor checks
--
	
check_arinc653_virtual_processors () <=
	** "Virtual Processors are in processors" **
	forall (vp : virtual_processor) . true => 
	             (exists (cpu : processor) . parent(vp) = cpu) and
	             (has_property (vp, ARINC653::Partition_Identifier)) and
	             (has_property (vp, ARINC653::Partition_Name))

--
-- Process checks
--
check_arinc653_processes() <=
   ** "All processes are bound to a memory segment and a virtual processor" **
   forall (p : process) . true =>  check_arinc653_process_memory (p) and 
                                   check_interfaces (p) and
                                   check_tasks (p) and 
                                   check_arinc653_process_virtual_processor (p)

check_tasks (p : process) <=
   ** "Check tasks from process " p **
   forall (thr : thread) . (parent (thr) = p) => check_interfaces (thr) 

check_interfaces (comp : component) <=
	** "Check that component " comp " declares all necessary properties on its ports" **
	forall (poevent : features (comp)) . is_event_port (poevent) and (direction(poevent) = "in") => has_property (poevent, ARINC653::Queueing_Discipline)
	and
	forall (pononevent : features (comp)) . (is_event_port (pononevent) = false) and (direction(pononevent) = "in") => has_property (pononevent, ARINC653::Sampling_Refresh_Period)


check_arinc653_process_memory (p : process) <=
	** "Check that process " p " is associated with a memory" **
	exists (segment : memory) . (is_bound_to (p, segment)) and check_arinc653_memory_segment (segment)

check_arinc653_memory_segment (segment : memory) <=
	** "Check that the memory segment " segment " is contained in a memory" **
	exists (mem : memory) . (parent(segment) = mem)

check_arinc653_process_virtual_processor (p : process) <=
	** "Check that process " p " is associated with a virtual processor" **
	exists (runtime : virtual_processor) . (is_bound_to (p, runtime))		
\end{lstlisting}


\section{Conclusion}
This tutorial explains the core principles of RESOLUTE and how to use
the language to validate an architecture. If you have any question
or experience issue, please visit the project page 
\footnote{project page on github on \url{https://github.com/smaccm/smaccm}}
and submit an issue on the issue tracker.

\end{document}
